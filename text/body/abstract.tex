\begin{abstract}
Within asexual populations, heritable mutator alleles that globally elevate an organism's mutation rate impose a complex evolutionary trade-off: on one hand, accelerating adaptation but, on the other, increasing mutational load.
Aspects of existing theory suggest that large asexual populations should generally favor the fixation of mutator alleles due to ``hitch-hiking'' effects associated with beneficial mutation discovery.
However, this outcome is often not the case in natural populations, and mechanisms preventing mutator allele fixation in such large asexual populations are less completely understood.
In this study, we investigate how scenarios with few beneficial mutations available (i.e., limited adaptive potential) affect mutator dynamics across continuums of population size and spatial structure.
Experiments utilizing stochastic agent-based simulations confirm that when adaptive potential is restricted, large population size disfavors mutator allele fixation --- an opposite effect compared to large populations with unlimited adaptive potential.
Moreover, we find spatial population structure to act synergistically in suppressing mutator fixation within large populations, but only when the mutator allele is initially rare and adaptive potential is limited.
In contrast, under otherwise-equivalent well-mixed conditions where mutators are initially rare, population size more consistently promotes mutator fixation.
These findings underscore the critical role of adaptive potential in shaping the evolution of mutation rates in asexual populations, with implications for understanding and managing hypermutation-associated phenomena in both laboratory and real-world contexts.
We also describe technical aspects of the work in harnessing the 850,000 processor Cerebras Wafer-Scale Engine hardware to enable large-scale agent-based simulations, which allowed up to a $111{,}091\times$ speedup over single-core CPU execution and a $294\times$ speedup over GPU execution.
\end{abstract}

%%
%% The code below is generated by the tool at http://dl.acm.org/ccs.cfm.
%% Please copy and paste the code instead of the example below.
%%
\begin{CCSXML}
<ccs2012>
 <concept>
  <concept_id>00000000.0000000.0000000</concept_id>
  <concept_desc>Do Not Use This Code, Generate the Correct Terms for Your Paper</concept_desc>
  <concept_significance>500</concept_significance>
 </concept>
 <concept>
  <concept_id>00000000.00000000.00000000</concept_id>
  <concept_desc>Do Not Use This Code, Generate the Correct Terms for Your Paper</concept_desc>
  <concept_significance>300</concept_significance>
 </concept>
 <concept>
  <concept_id>00000000.00000000.00000000</concept_id>
  <concept_desc>Do Not Use This Code, Generate the Correct Terms for Your Paper</concept_desc>
  <concept_significance>100</concept_significance>
 </concept>
 <concept>
  <concept_id>00000000.00000000.00000000</concept_id>
  <concept_desc>Do Not Use This Code, Generate the Correct Terms for Your Paper</concept_desc>
  <concept_significance>100</concept_significance>
 </concept>
</ccs2012>
\end{CCSXML}

\ccsdesc[500]{Do Not Use This Code~Generate the Correct Terms for Your Paper}
\ccsdesc[300]{Do Not Use This Code~Generate the Correct Terms for Your Paper}
\ccsdesc{Do Not Use This Code~Generate the Correct Terms for Your Paper}
\ccsdesc[100]{Do Not Use This Code~Generate the Correct Terms for Your Paper}

%%
%% Keywords. The author(s) should pick words that accurately describe
%% the work being presented. Separate the keywords with commas.
\keywords{Do, Not, Us, This, Code, Put, the, Correct, Terms, for,
  Your, Paper}
%% A "teaser" image appears between the author and affiliation
%% information and the body of the document, and typically spans the
%% page.
% \begin{teaserfigure}
%   \includegraphics[width=\textwidth]{sampleteaser}
%   \caption{Seattle Mariners at Spring Training, 2010.}
%   \Description{Enjoying the baseball game from the third-base
%   seats. Ichiro Suzuki preparing to bat.}
%   \label{fig:teaser}
% \end{teaserfigure}

\received{20 February 2007}
\received[revised]{12 March 2009}
\received[accepted]{5 June 2009}
