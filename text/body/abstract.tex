\begin{abstract}
Within evolving microbial populations, genes that elevate mutation rate impose a fundamental trade-off: on one hand, increasing harmful mutations among offspring but, on the other, allowing more opportunities for rare beneficial mutations.
Existing single-CPU agent-based simulation work suggests that increased population size should generally favor the fixation of mutator alleles due to ``hitch-hiking'' effects associated with beneficial mutation discovery.
However, in contrast to this expectation, this outcome is often not the case in large asexual populations found in nature.
To address this knowledge gap, we leveraged the 850,000 processor Cerebras Wafer-Scale Engine (WSE) to increase simulation scale up to 1.5-billion-agent populations.
In benchmarks, WSE provided $294\times$ speedup over GPU and $111{,}091\times$ speedup over single-core CPU execution.
Among other results, our experiments indicate that limitation of adaptive potential (i.e., few beneficial mutations available) can produce a tertiary regime where fixation of mutator alleles becomes disfavored at very large population sizes.
\end{abstract}

%%
%% The code below is generated by the tool at http://dl.acm.org/ccs.cfm.
%% Please copy and paste the code instead of the example below.
%%
\begin{CCSXML}
<ccs2012>
   <concept>
       <concept_id>10010405.10010444</concept_id>
       <concept_desc>Applied computing~Life and medical sciences</concept_desc>
       <concept_significance>500</concept_significance>
       </concept>
   <concept>
       <concept_id>10010147.10010341</concept_id>
       <concept_desc>Computing methodologies~Modeling and simulation</concept_desc>
       <concept_significance>500</concept_significance>
       </concept>
   <concept>
       <concept_id>10010520.10010521.10010528.10010535</concept_id>
       <concept_desc>Computer systems organization~Systolic arrays</concept_desc>
       <concept_significance>100</concept_significance>
       </concept>
 </ccs2012>
\end{CCSXML}

\ccsdesc[500]{Applied computing~Life and medical sciences}
\ccsdesc[500]{Computing methodologies~Modeling and simulation}
\ccsdesc[100]{Computer systems organization~Systolic arrays}
%%
%% Keywords. The author(s) should pick words that accurately describe
%% the work being presented. Separate the keywords with commas.
\keywords{agent-based modeling and simulation, evolutionary biology, high-performance computing, wafer-scale computing, Cerebras Wafer-Scale Engine}
%% A "teaser" image appears between the author and affiliation
%% information and the body of the document, and typically spans the
%% page.
% \begin{teaserfigure}
%   \includegraphics[width=\textwidth]{sampleteaser}
%   \caption{Seattle Mariners at Spring Training, 2010.}
%   \Description{Enjoying the baseball game from the third-base
%   seats. Ichiro Suzuki preparing to bat.}
%   \label{fig:teaser}
% \end{teaserfigure}

\received{18 August 2025}
% \received[revised]{12 March 2009}
% \received[accepted]{5 June 2009}
